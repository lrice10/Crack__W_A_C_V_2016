\documentclass[10pt,twocolumn,letterpaper]{article}

\usepackage{styles/wacv}
\usepackage[utf8]{inputenc}
\usepackage{times}
\usepackage{epsfig}
\usepackage{graphicx}
\usepackage{amsmath}
\usepackage{amssymb}

\usepackage{natbib}
\usepackage{graphicx}

\usepackage{epstopdf}
\usepackage{array}
\usepackage{multirow}
\usepackage{rotating}
\usepackage{color}
\usepackage{tabulary}
\usepackage{caption} 
\usepackage[pagebackref=true,breaklinks=true,colorlinks,bookmarks=false]{hyperref}

\newcommand\comment[1]{\textcolor{red}{{\tt #1}}}
\newcommand\vect[1]{\boldsymbol{#1}}
\newcommand{\argmax}[1]{\underset{#1}{\operatorname{argmax}}}

\wacvfinalcopy % *** Uncomment this line for the final submission

\def\wacvPaperID{121} % *** Enter the wacv Paper ID here
\def\httilde{\mbox{\tt\raisebox{-.5ex}{\symbol{126}}}}

\ifwacvfinal\pagestyle{empty}\fi

\begin{document}
\title{WACV 16 - Crack Detection Paper}

\author{
Authors$^\ast$
\begin{tabular}{c c}
$^\ast$~Univ. of North Carolina at Charlotte \\
Charlotte, NC 28223\\
\end{tabular}\\
{\tt \small author@uncc.edu  } }
\maketitle

\begin{abstract}
[Abstract here]
\end{abstract}



\section{Introduction}
Periodic inspection is important to ensure the safety of nuclear power plant components. Manually inspecting and evaluating 100+ hours of video for rarely occurring cracks is a tedious process. However, automatic inspection is challenging as the images often contain highly textured area including weld and concrete surface which causes fragmented and noisy segmentations.

Koch et. al \cite{koch2015review_survey} review the different methods for detecting defects civil infrastructures. They summarize the different methods of crack detection in the domains of reinforced concrete bridges, precast concrete tunnels, underground concrete pipes, and asphalt pavements. They generalize that all the methods usually have two steps, pre-processing, and crack-identification. The pre-processing involves extracting features from lines or edges. They categorize crack-identification step into 3 categories, threshold-based, model-based, and pattern-based approaches. 

Recently, several methods for automatic crack detection have been proposed to address the challenges of non-crack noises and crack intensity discontinuity. Segment extending [\cite{SEG_EXTEND}], tensor voting [\cite{ZouTensorVoting}], and extending a saliency map [SALIENCY] have been applied to connect disconnected cracks. A block SVM classifier using soft hough features was used to detect cracks in noisy images [\cite{HTF_SVM}]. Seed growing [\cite{SEED_GROW}] and crack saliency [\cite{SALIENCY}] using statistical features have also been applied to reduce detecting noise as crack. Jahanshahi et al. segment line-like segments using morphological operations, extract geometric features from the segments and then use a neural network classifier to determine if the segment is a crack [\cite{jahanshahi2013innovative}].

Hand crafted features and machine learning classifiers have been used to detect cracks. A block SVM classifier using soft hough features was used to detect cracks in noisy images HTF-SVM. Oliveira et. al develop a system to detect road cracks using an unsupervised learning method \cite{Oliveira2013_road_blockClassification} \cite{Oliveira2014_Road_Classify_ICIP}.  Gabor features along with an adaboost classifier are used to detect road cracks \cite{Medina2014_Road_GaborAdaboost}.  Chanda et. al  detect cracks in highly textured bridges using texture features and SVM \cite{chanda2014_bridges_textureSVM}. Kapela et. al. detect asphalt using HOG and a SVM classifier \cite{Kapela2015_Road_HOG_SVM}. Prasanna et. al. divide the image into blocks, detect lines using RANSAC in each block, extract gradient and intensity features from the block at different scales and then classify the block as crack or not using a classifier \cite{Prasanna2014_Bridges_classifierRANSAClines}. Jahanshahi et al. segment line-like segments using morphological operations, extract geometric features from the segments and then use a neural network classifier to determine if the segment is a crack \cite{jahanshahi2013innovative}. 

Zou et. al detect defects in weld pipe inspection videos by using a Kalman filter to detect the continuity of the defect motion as compared to just noise detections.   \cite{Zou2015_weldDefect_Kalman}


Convolutional Neural Networks and modern GPUs have brought great success in object classification in recent years with the monumental success of Alexnet in the 2012 ImageNet competition \cite{krizhevsky2012imagenet}.  The deeper but less parameters GoogLeNet CNN performed the best in the 2014 ImageNet Competition \cite{googlenet2014going}. Instead of hand crafting features, the CNN learns features that are useful to the object classification task. The Deep Learning CNNs learned in the multi-class object classification can be fine tuned to other tasks such as face detection \cite{farfade2015multi} and style classification \cite{karayev2013styleRecognizing}.

In this paper, we propose to improve the detection of cracks by (1) fine-tuning the GoogLeNet convolutional neural network for crack block classification, and (2) group the detected crack patches by fitting planes in the spatial-temporal space (x,y,time). Testing on 17 real videos demonstrates accuracy of XX TP and XX FP rates which is XX\% improvement over prior method. Testing of 42 real images demonstrates 38% improvement over prior method. Note that the evaluation of the method is conducted at the classification of image level rather than the segmentation at pixel-level.



\section{Overview}

A GoogLeNet CNN is finetuned trained to distinguish between crack and non-crack patches
For each frame in video, the frame is  divided into 70 overlapping blocks. Each patch is ran through the CNN classifier. For frame based detection, If any patch is classified as crack, the frame is classified as having crack in it.
All the blocks that are classified as crack in a single video are put into 3D space. RANSAC plane fitting finds potential crack frame interval groups.
Each plane is classified as crack or not by using a simple threshold method. 


\subsection{CNN training}
           A GoogLeNet is finetuned trained to determine if an 224x224 image patch has a crack inside it or not. Approximately 1.5 million 224x224 image patches are extracted from 16 videos for each training set. Crack patches are selected from manually pixel labeled ground truthed crack frames with a crack pixel in the center of the patch. Additional crack patch samples are added by rotating the initial crops by 90 degrees.  Negative non-crack patches are taken from frames that do not have crack. Pixels with greater morph response in the non-crack frames are given higher probability of being selected as a training patch. An equal ratio of positive and negative samples is used. The mean RGB color of ImageNet is subtracted from each patch.
           In training, the CNN weights are initialized from the GoogLeNet that is available at the Caffe model zoo. The final inner product layer is the layer that is initialized from scratch. We use caffe \cite{caffe} to finetune train the GoogLeNet.  The blob learning rate coefficients of the final inner product layer are 10 times higher than the learning rate coefficients from the layers initialized from GoogLeNet.  The base learning rate. is set to  $0.001$,   gamma to $0.1$, momentum $0.9$, and weight decay $0.0005$.  We use a training batch size of 128 for each iteration for a total of 25,000 iterations.  A softmax classifier is attached to the final layer for classification.
            
\subsection{Crack block detection}
            For each frame in a video, the frame is divided into 70 overlapping 224x224 blocks. Adjacent blocks contain about 75\% overlap in area. The ImageNet mean RGB is subtracted from each block and then it is run through the CNN and classified as either crack or not.

\subsection{Crack detection through spatial temporal grouping}
            Sometimes weld, grind, and scratches can be detected sporadically throughout the video and this can sometimes bring false alarms at the frame level. Crack blocks appear sequentially in time, and their movement through time follows the movement of the camera.  Grouping the detections temporally helps filter out the noisy sporadic false detections and helps identify a series of detected crack frames as single crack instance.
All the crack block detections in a video, are inserted into a 3D point cloud from their (x,y, t). X, and Y, are the coordinates of the center of the block in the frame, and t is the frame number that they appear in.   From the 3D point cloud, RANSAC plane fitting is run iteratively to find planes.  For each plane found, the inlier points of the found plane are removed, and RANSAC plane fitting is run again. This is done until no more planes are found.  For the settings for RANSAC plane fitting, a reference vector of [0, 0, 1] is used. It points approximately parallel to normal vectors of the desired planes with max angle offset between the two vectors of 15 degrees. Each inlier point of the plane has a maximum distance from the surface of the plane of about 500.  

\subsection{Classification of found planes}
         For each found plane,  2 features are extracted. The first feature is the length in frames that the points in the plane occupy.  The second feature is the percentage of frames in the interval that are detected as crack. Detections from the training set are used to find the mean and standard deviation(sigma) for each of these two features.  The threshold for each feature is computed as the mean - 2sigma.    A plane  where these 2 values are both greater than the thresholds, are kept as classified crack planes.  

\section{Experimental Evaluation}
    \subsection{Dataset}
    We have collected 17 videos of the synthetically generated cracks of stainless steel plates containing laser machined flaws(cracks) (Figure 6.) The videos are captured by Ahlberg MegaRad L Camera at a distance which was approximately 4-6 inches away from the plates.
            There are 45 unique cracks from all the videos. Cracks can vary in length, thickness, and distinguishability. 
The videos are about 10,000 frames long for each video. The videos have various weldings, scratches, and grind marks.

\subsection{Evaluation metric}
\paragraph{Leave one out evaluation}
      For each video, the training set for the CNN and plane classification, is from the other 16 videos.

\paragraph{Frame-level}
      For each video, frames are marked as crack if any part of a crack appears in the frame. When evaluating the cracks,  crack groups are created from frames that are consecutively labeled as cracks.  In each crack group, the first 25\% of the frames of the crack group and the last 25\% of the crack group are not used in the frame based evaluation and are labeled as “don’t care”. This is because at these times only a small part of the crack is visible, and the crack usually can not be clearly distinguished.
      TP if any block in the frame is detected as crack and the ground truth is labeled crack.
      FP if any block in the frame is detected as crack and the ground truth is labeled non-crack.
      TN if no blocks in the frame are detected as crack and the ground truth is labeled non-crack.
      FN if no blocks in the frame are detected as crack and the ground truth is labeled crack.

We use TPR, FPR, and F1-Score to evaluate performance at the frame level. 
TPrate = TPs / (TPs + FNs);
FPrate = FPs / (FPs + TNs);

F1score =(2 * TPs) / (2 * TPs + FPs + FNs);

\paragraph{crack-level}
     We also evaluate at the crack camera pass level. We use an overlap strategy based on \cite{HooverGoldgofpaper}.
     GT Crack groups are created from frames that are consecutively labeled as cracks. GT non-crack groups are created from consecutive frames labeled as non-cracks.  For each crack detected plane interval and GT Crack Group combination, the number of frames of overlap is computed and then the $overlap-percentage = \min( percent-frame-Overlap-with-GT, percent-frame-Overlap-with-detection)$ is computed.   Each GT Crack group is matched to the crack detection group with the highest overlap-percentage if the $overlap-percentage >= 0.1$. This is then counted as a TP.  Each GT Crack Group can only be matched to one crack detection group. 
FNs are the number GT Crack groups that do not get matched to a crack group detection. 
FPs are the number of crack group detections that do not get matched to a GT Crack group.
TNs are the number GT non-crack groups that get matched to non-crack groups from detection using similar overlap criteria as TPs. 



\subsection{Accuracy}
TPR, FPR, F-score (include 4” dataset results only)


\section{Conclusion and Future Work}
[Conclusion Here]


\bibliographystyle{styles/ieee}
\bibliography{references}

\end{document}

\documentclass{article}
