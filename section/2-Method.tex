%% ------------------------------------------------
%          Challenging Parts figure (?):
%% ------------------------------------------------
\begin{figure*}

    \begin{centering}
        \includegraphics[width=0.6\textwidth]{Images/NoVisualHolder.png}
        
        \caption{Here is an figure representing challenges and/or examples}
        \label{fig:FigChallenges}
    \end{centering}
    
\end{figure*}
%% ------------------------------------------------



\section{Overview}

    \noteError[inline]{[Lance:] Waiting on replies about method questions. Some of them are critical for formulating a concistent and accurate notation and story.}
    \begin{enumerate}
        \item A GoogLeNet CNN is finetuned trained to distinguish between crack and non-crack patches.
        \item For each frame in video, the frame is:
            \begin{enumerate}
                \item Divided into 70 overlapping patches
                \item CNN classifies each patch
                \item Frame is classified as having crack in it if contains any crack patches.
            \end{enumerate}
        \item All the blocks that are classified as crack in a single video are put into 3D space and RANSAC plane fitting occurs.
        \item Each plane is classified as crack or not by using a simple threshold method.
    \end{enumerate}


    %% ------------------------------------------------
    %                 Examples Figure (?):
    %% ------------------------------------------------
    \begin{figure*}
    
        \begin{tabular}{c|c}
            \includegraphics[width=\columnwidth]{Images/NoVisualHolder.png} & \includegraphics[width=\columnwidth]{Images/NoVisualHolder.png}
        \end{tabular}
        
        \caption{Here is an figure of good/bad performance examples}
        \label{fig:FigExample}
        
    \end{figure*}
    %% ------------------------------------------------

    \subsection{CNN training}
                A GoogLeNet is finetuned trained to determine if an 224x224 image patch has a crack inside it or not. Approximately 1.5 million 224x224 image patches are extracted from 16 videos for each training set. Crack patches are selected from manually pixel labeled ground truthed crack frames with a crack pixel in the center of the patch. Additional crack patch samples are added by rotating the initial crops by \noteImprove{[Lance$>$Steve:] Just +90, or +/- 90?}90 degrees.  Negative non-crack patches are taken from frames that do not have crack. Pixels with greater morph response \noteImprove{[Lance$>$Steve:] I'm not sure what "morph response" means.} in the non-crack frames are given higher probability \noteImprove{[Lance$>$Steve:] How is this probability calculated? Weighted Sampling?} of being selected as a training patch. An equal ratio of positive and negative samples is used. The mean RGB color of ImageNet is subtracted from each patch \noteImprove{[Lance$>$Steve:] The ImageNet mean image or the mean image calculated by ImageNet from the supplied patch examples?}.
                
                In training, the CNN weights are initialized from the GoogLeNet that is available at the Caffe model zoo. The final inner product layer is the layer that is initialized from scratch. We use caffe \cite{jia2014} to finetune train the GoogLeNet.  The blob learning rate coefficients of the final inner product layer are 10 times higher than the learning rate coefficients from the layers initialized from GoogLeNet.  The base learning rate. is set to  $0.001$,   gamma to $0.1$, momentum $0.9$, and weight decay $0.0005$.  We use a training batch size of 128 for each iteration for a total of 25,000 iterations.  A softmax classifier is attached to the final layer for classification.
                
                
    \subsection{Patch-Level Crack Detection}
                For each frame in a video, the frame is divided into 70 overlapping 224x224 \noteImprove{[Lance$>$Steve:] Why this particular size? ImageNet constraints and/or choice? Is it so that you get 70 patches?} blocks. Adjacent blocks contain about 75\% overlap in area. The ImageNet mean RGB is subtracted from each block and then it is run through the CNN  \noteImprove{[Lance$>$Steve:] Binary output, or real value that is thresholded?}  and classified as either crack or not.
    
    
    \subsection{Spatial-Temporal Patch grouping}
    \label{grouping}
                Sometimes weld, grind, and scratches can be detected sporadically throughout the video and this can sometimes bring false alarms at the frame level. Crack blocks appear sequentially in time, and their movement through time follows the movement of the camera.  Grouping the detections temporally helps filter out the noisy sporadic false detections and helps identify a series of detected crack frames as single crack instance.
                
                All the crack block detections in a video, are inserted into a 3D point cloud from their (x,y, t). X, and Y, are the coordinates of the center of the block in the frame, and t is the frame number that they appear in.   From the 3D point cloud, RANSAC plane fitting is run iteratively to \noteImprove{[Lance$>$Steve:] What is the stopping condition?} find planes.  For each plane found, the inlier points of the found plane are removed, and RANSAC plane fitting is run again \noteImprove{[Lance$>$Steve:] Could a point belong to multiple plans? If so, is there a chance of none of the plans satisfying the thresholds because of the order in which the points were removed?}. This is done until no more planes are found.  For the settings for RANSAC plane fitting, a reference vector of [0, 0, 1] is used. It points approximately parallel to normal vectors of the desired planes with max angle offset between the two vectors of 15 degrees. Each inlier point of the plane has a maximum distance from the surface of the plane of about 500.  
    
    
    \subsection{Classification of found planes}
            For each plane found in \ref{grouping}, 2 features are extracted:
            \begin{enumerate}
                \item  $length$ - length in frames that the points in the plane occupy
                \item $percent$ - percentage of frames in the interval that are detected as crack
            \end{enumerate}
            Detections from the training set are used to find the mean and standard deviation (sigma) for each of these two features.  The threshold for each feature is computed as the \noteImprove{[Lance$>$Steve:] Any rational that could be supplied?} mean - 2sigma. A plane  where these 2 values are both greater than the thresholds, are kept as classified crack planes.  
         

