



% \marginnote{This is a margin note using the geometry package, set at 
% 3cm vertical offset to the line it is typeseted.}[3cm]

\section{Introduction}
    Structural components of nuclear power plants require periodic inspection to ensure safe uninterrupted operation and avoid structural damage. The structural damage can be inspected by monitoring the videos recorded along the concrete wall? of the power plan and detecting any development of cracks or crack-like defects. Since cracks are rare occurrences during the inspection, manually monitoring and evaluating 100+ hours of video in order to detect them is a highly tedious process. To alleviate time and effort of the human inspector, automatic methods of crack detection are of great interest in the applications of computer vision as well as (in infrastructure inspection?).   However, automatic inspection is challenging as the images of these videos often contain highly textured area including weld and concrete surface which causes fragmented and noisy segmentations for any automatic method.
 
\begin{figure} [ht]
\begin{centering}
\begin{tabular}{c}
\includegraphics[width=\columnwidth]{Images/NoVisualHolder.png}
\end{tabular}
\caption{Eye grabbing image that sparks interest and confidence.}
\label{fig:FigTease}
\end{centering}
\end{figure}
 
    Many automatic methods for crack detection have been developed during the last few years. In a survey paper by Koch et. al \cite{Koch:2015} different methods for detecting defects civil infrastructures  were discussed extensively. The authors summarize the different methods of crack detection in the domains of reinforced concrete bridges, precast concrete tunnels, underground concrete pipes, and asphalt pavements. They generalize that all the methods usually have two steps, pre-processing, and crack-identification. The pre-processing involves extracting features from lines or edges. They categorize crack-identification step into 3 categories, threshold-based, model-based, and pattern-based approaches. 

\begin{figure*}
\begin{centering}
\includegraphics[width=0.6\textwidth]{Images/NoVisualHolder.png}
\caption{This figure represents the method at an abstract level in a brief glance.}
\label{fig:FigMain}
\end{centering}
\end{figure*}

    Recently, several methods for automatic crack detection have been proposed to address the challenges of non-crack noises and crack intensity discontinuity. Segment extending \cite{Liu2008}, tensor voting \cite{Zou2012}, and extending a saliency map \cite{Xu2013} have been applied to connect disconnected cracks. A block SVM classifier using soft hough features was used to detect cracks in noisy images \cite{Hu2010}. Seed growing \cite{Li2011} and crack saliency \cite{Xu2013} using statistical features have also been applied to reduce detecting noise as crack. Jahanshahi et al. segment line-like segments using morphological operations, extract geometric features from the segments and then use a neural network classifier to determine if the segment is a crack \cite{jahanshahi2013}.

    Machine learning classifiers have been used to detect cracks from its hand-crafted features. A block SVM classifier using soft hough features was used to detect cracks in noisy images HTF-SVM. Oliveira et. al develop a system to detect road cracks using an unsupervised learning method \cite{Oliveira2013} \cite{Oliveira2014}.  Gabor features along with an adaboost classifier are used to detect road cracks \cite{Medina2014}.  Chanda et. al  detect cracks in highly textured bridges using texture features and SVM \cite{chanda2014}. Kapela et. al. detect asphalt using HOG and a SVM classifier \cite{Kapela2015}. Prasanna et. al. divide the image into blocks, detect lines using RANSAC in each block, extract gradient and intensity features from the block at different scales and then classify the block as crack or not using a classifier \cite{Prasanna2014}. Jahanshahi et al. segment line-like segments using morphological operations, extract geometric features from the segments and then use a neural network classifier to determine if the segment is a crack \cite{jahanshahi2013}. 

    Zou et. al detect defects in weld pipe inspection videos by using a Kalman filter to detect the continuity of the defect motion as compared to just noise detections.   \cite{Zou2015}

    Convolutional Neural Networks and modern GPUs have brought great success in object classification in recent years with the monumental success of Alexnet in the 2012 ImageNet competition \cite{krizhevsky2012}.  The deeper but less parameters GoogLeNet CNN performed the best in the 2014 ImageNet Competition \cite{szegedy2014}. Instead of hand crafting features, the CNN learns features that are useful to the object classification task. The Deep Learning CNNs learned in the multi-class object classification can be fine tuned to other tasks such as face detection \cite{farfade2015} and style classification \cite{karayev2013}.

    In this paper, we propose to improve the detection of cracks by (1) fine-tuning the GoogLeNet convolutional neural network for crack block classification, and (2) group the detected crack patches by fitting planes in the spatial-temporal space (x,y,time). Testing on 17 real videos demonstrates accuracy of XX TP and XX FP rates which is XX\% improvement over prior method. Testing of 42 real images demonstrates 38\% improvement over prior method. Note that the evaluation of the method is conducted at the classification of image level rather than the segmentation at pixel-level.




